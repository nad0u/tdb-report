%% Introduction


\chapter{Introduction}
Since several years, \textit{Cloud Computing} gained in popularity, mostly because it allows companies and individuals to store or analyze large amount of data with reduced costs. 
In the domain of \textit{Cloud Computing}, most individuals are only using softwares or services made available by third parties. 
As examples, we can mention Gmail from Google, which lets you send and receive e-mails, or Drive from Google for storing and sharing your documents. 
The webmail services have been around for several years and are using the \textit{Cloud Computing}. 
Indeed, the users do not need to worry about the installation or the update of the server. 
All they need to do is to use the available service. In the case of webmail, the users do not need to know where and how their e-mails are stored, as long as they are stored somewhere accessible remotely. This is all the interest about \textit{Cloud Computing} for the end users.

Concerning the companies, they are more and more that want to store and analyze their data to better understand their clients. 
The amount of these data can be counted in terabytes or even petabytes. 
To deal with such an enormous amount of data, powerful systems are needed. 
However, all companies, especially the startups, have not the means to invest in new systems, because it will cost too much (machines, installation, maintenance, etc). 
Moreover, if the amount of data or computations a company deals with grows too fast, it will take too long to put in place new systems, and thus it will take longer to analyze all the data. 
\textit{Cloud Computing} helps by giving the possibility to quickly and easily add more storage, more memory or even more CPUs depending the need. 
The basic principal of \textit{Cloud Computing} is to provide hardware resources easily and rapidly over the Internet.

From these few examples, we can see how convenient \textit{Cloud Computing} can be, either for companies or individuals. 
This is why the market has been growing more and more these last few years. 
Among the major players in the cloud, we can mention Amazon and Google. 
Amazon is the most known for its cloud services, \textit{Amazon Web Services (AWS)}, which let the user create a server (i.e. a virtual machine) in a few mouse clicks and begin to develop on it.  
It is a service on demand where the clients only need to pay what they consume (i.e. computing power). 
Moreover, the company using this service does not need to worry about the machines maintenance, which is done by the service provider.



\section{Motivation and method}
\textit{Cloud Computing} is very interesting because one can easily add computing power to deal with a growing project. 
The goal of this bachelor work was to experience with open source cloud computing solutions. 
OpenStack, one of the biggest open source project (and free) out there was chosen for this bachelor work. 
It offers the possibility to create its own cloud infrastructure, which can then be managed and used through a web interface.

At first, OpenStack will be installed and configured on three chosen physical machines. 
Once everything is working correctly, OpenStack will be benchmarked with the help of OLTP-Bench, another open source project. 
OpenStack will be benchmarked with different configurations and the same benchmark will run on the physical machines as a point of comparison.


% \section{Method}
