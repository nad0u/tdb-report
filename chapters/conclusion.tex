%!TEX root = ../main.tex
%% Conclusion


\chapter{Conclusion}
In this chapter summarizes the findings of this bachelor work, discusses the encountered problems and outlines potential future work.


%%--------------- section
\section{Findings}
The goal of this bachelor work was to get to know OpenStack, an open source cloud infrastructure project, and benchmark it.
As a personal note, no knowledge in the field of \textit{Cloud Computing} was there beforehand, especially in how everything works in details.
The only knowledge was those of an end-user just using data storage services like Dropbox, and being happy that it works well.
This bachelor work was a great opportunity to deepen these knowledge, by getting to know how everything works behind the scene, or how to setup a cloud infrastructure and being able to use it.

Coming back to the main topic, in order to be able to benchmark OpenStack, virtual machines created with OpenStack were benchmarked with the help of the OLTP-Bench project, an open source project supporting several types of benchmarks including TPC-C.
The benchmark was first run on a dedicated physical machine, whose results are used for comparison with the other benchmarks.
In order to benchmark OpenStack, the OLTP-Bench was run on virtual machines with different flavors and with or without logical storage attached to them.
In Phase 2 (virtual machines without storage), we obtained quite good performance results, but OpenStack could not outperform the the physical machine from Phase 1.
In Phase 3 (virtual machines with storage), we obtained very good results, even better than the physical machine.
From this we can conclude that attaching volumes to virtual machines can really improve the performance, and that OpenStack manages well the volumes.
In Phase 4 (testing isolation), the results obtained are not quite good, this is caused by the fact that both virtual machines are accessing the same hard drive hosted on \texttt{compute2}.

From these three phases, we can conclude that OpenStack is quite performant and can even perform better than physical machines.
Also, we observed that in general there is little difference in performance when comparing \textit{large} and \textit{physical} flavor virtual machines.
To be sure of this fact, it could be interesting to run the same benchmark on a virtual machine with \textit{small} or \textit{tiny} flavors.
A general observation, is that when the scale factor increases, the throughput of the system decreases.

Finally, OpenStack is a very interesting open source project that is gaining in popularity as a free IaaS.
Its performance is quite good, which is important in \textit{Cloud Computing} as its goal is to deliver computational power on demand.
The web interface is very intuitive and facilitates the management of the cloud infrastructure after a successful setup.







%%--------------- section
\section{Problems}
During this bachelor work, several problems arose at different stages. 
At first, we encountered problems when installing OpenStack on the different physical machines.
The process is very long as there is no ``one click'' setup as one could expect with a simple software installation.
OpenStack relies on configuration files after installing the different components.
This is where problems arose, as sometimes there were some little errors in the official documentation.
For example, it was specified that only the hostname of a machine can be used instead of giving the IP address for certain configuration files; but in reality, it was necessary to use the IP address instead. 
This type of error is quite minor (it was encountered when setting up VNC) and is easily solved if the status of OpenStack is checked at each step of the installation.
We also encountered network problems during the installation: at some point, it was possible to create instances but there were no IP assigned to them, and thus it was impossible to SSH the virtual machines.
This was solved by reserving IP addresses from the university network.

Concerning the benchmarks, it took a few attempts in order to find a good configuration to run the tests.
At first we only tried the default values\footnote{\url{https://github.com/oltpbenchmark/oltpbench/blob/master/config/sample_tpcc_config.xml}, 21.05.2015}, and observed that after each run the throughput was increasing.
So we did not know how many runs it was necessary to perform in order to have meaningful data.
We tried several configurations by increasing the duration and the scale factor.
We finally decided to use the values presented in the Listing \ref{lst:lst_oltp_config}.


%%--------------- section
\section{Future work}
With the different tests performed, we obtained quite useful insights regarding the OpenStack performance.
Below there are some other suggestions that could be used to obtain more meaningful results.

First, as it has been shown in Phase 4, it could be more interesting to use a machine dedicated to volume management in order to test isolation.
Better performance results might be observed.
One can also try to run the same benchmark as in Phase 4, without using volumes.
But as Phase 3 shown that better performance can be achieved by attaching volumes to the virtual machines, it seems logical to test isolation with volume also.

Then, other tests (for all Phases) can be performed by increasing the duration of a run. 
It could be better to extend the duration to 40 minutes for example.
Also, different number of terminals (20 and 80) can be used to see their impact on the performance.





