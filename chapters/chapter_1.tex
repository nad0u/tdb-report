%!TEX root = ../main.tex
%% Chapter 1 - Cloud Computing


\chapter{Cloud Computing}
During the last decade, \textit{Cloud Computing} technology got a lot of traction, but many people do not know what this technology stands for exactly. 
In fact, we all use cloud computing since several years without actually knowing it. One prominent example is webmail, a service which gives a possibility to send and receive e-mails.
More formally, \textit{Cloud Computing} offers users a possibility to do something (i.e. writing a text document) on the web without the need to have any software installed locally, everything is being done on a remote server. 
This is exactly what is being offered to the users by webmail services.

When talking about the \textit{cloud}, one must make the difference between private, hybrid and public clouds. 
A private cloud allows to have the advantages of the cloud, but it is restrained to the organization, meaning that it is not exposed to the Internet and that all the hardware resources (i.e. servers) reside inside the organization. 
Thus the services are accessible only when the employees are connected to the internal network of the organization. 
On the other hand, a public cloud is exposed to the Internet, meaning that we can access it from everywhere as long as we have an Internet connection. 
A good example is \textit{Dropbox}\footnote{\url{https://dropbox.com}} service, which offers the possibility to store data in the cloud. 
In the middle, there is the hybrid cloud, which is a mix of the previous two types of clouds.

In addition to the three types presented earlier, there exist several cloud models.
The three most important ones are IaaS, PaaS and SaaS \rp{give non-acronym versions first, probably make a list.}, which will be presented in the next sections. 
These proposed models are not always free, or in most cases they are free with a limited access to resources \rp{You're mixing a bit companies that provide services and models themselves, cloud model have nothing to do with pricing, they are just abstractions, I suggest to remove this and the next sentence.}. 
For intensive usage, the client (in general the company) must pay a monthly fee to the one provisioning the service.





%%--------------- subsection
\section{SaaS}
\textit{Software as a Service} (SaaS) model represents a remotely accessible software that is provided to the users. 
This type of software is also known as a hosted service. 
The software is accessible through Internet and the Web, so the user does not need to install anything on its own machine. 
Typical examples of such software are in businesses domain, such as Enterprise Resource Planning (ERP) and Customer Relationship Management (CRM) solutions. 
A more common piece of software is the webmail (e.g. Gmail), which lets users read and receive their e-mails without the need to install any mail client. 
Webmail services mentioned before also fall into this category. 
Another interesting example is the \textit{Google Drive}\footnote{\url{https://drive.google.com}} service, which lets users create and share documents, spreadsheets and presentation slides straight from a web browser. 
The user does not need to install any additional software to access it (except the web browser). 
As can be seen, this software is clearly intended for end users, who do not have to worry about security updates and all this stuff \rp{this is not scientific ;), try to rewrite somehow}, as this is entirely taken care by the one providing the applications.



%%--------------- subsection
\section{PaaS}
 \textit{Platform as a Service} (PaaS) model is intended for developing and deploying applications on an existing infrastructure.
Unlike SaaS, this model is designed for application developers so they do not need to install, configure or maintain any infrastructure (servers, network, etc.) as this is taken care by a PaaS provider. 
This way, PaaS fosters the work of developers by freeing them from system administration work. 
OpenShift, Heroku and Google App Engine are examples of PaaS providers \rp{create footnotes for every service mentioned as I did before}.
All these companies provide ready-to-use environments for developing applications.
The developers just need to choose the programming language in which they want to code and eventually also the framework and the database system, and everything is ready to start the development.



%%--------------- subsection
\section{IaaS}
Finally, \textit{Infrastructure as a Service} (IaaS) model means that infrastructure resources, such as CPU, memory, storage and servers, are provided to the client on demand. A better definition of an infrastructure could be:
\begin{quotation}
\textit{
IT infrastructure refers to the composite hardware, software, network resources and services required for the existence, operation and management of an enterprise IT environment. It allows an organization to deliver IT solutions and services to its employees, partners and/or customers and is usually internal to an organization and deployed within owned facilities.
}\cite{cjanssen14}
\end{quotation}

IaaS is not intended to final users, but to infrastructure architects, whose work is to ...\rp{please rewrite, network architects only do networks, not servers/memory/storage.}

Thanks to IaaS, companies do not have to spend money to buy new hardware to run their business, they can simply rent this hardware together with the computational resources they want, and it will be easy to add more computing power later if they need to and vice-versa. 
Moreover, they do not need to worry about the hardware itself as it is not owned by the company (in case of a public cloud). 
One of the most prominent companies proposing such services is Amazon \rp{add footnote}. 
Their service Amazon Elastic Compute Cloud (Amazon EC2) \rp{sentence looks unfinished}. 
In this work, we are going to use similar services offered by OpenStack, which will be presented in more details later.

Taking a step further, recently another type of model has risen: \textit{Metal as a Service} (MaaS) proposed by Canonical:
\begin{quotation}
\textit{A system that makes it quick and easy to set up the physical hardware on which to deploy complex scalable services, like Ubuntu’s OpenStack cloud infrastructure.}\footnote{\url{http://www.ubuntu.com/cloud/tools/maas}, 14.01.2015}
\end{quotation}

Contrary to IaaS, the goal of MaaS is to provide real servers on demand with the speed of a cloud instance. Then with the help of Juju\rp{add footnote + small descriptions, you need to introduce it to an unprepared reader} it is easier to deploy a IaaS like OpenStack on the hardware.






